\chapter{Matrix Factorization Methods}  
\label{chap:svd} 

% \section{The Goal}

Recall the brief introduction in Chapter \ref{chap:prologue}:  Let $A$
denote the ratings matrix, with the element in row $i$, column $j$
storing the rating of item $j$ given by user $i$.  Say the dimension of
$A$ is $u \times v$ matrix $A$.  We wish to find rank-$k$ matrices $W$
($u \times k$) and $H$ ($k \times v$) such that

\begin{equation}
\label{awh}
A \approx W H
\end{equation}

This is a form of dimension reduction, and typically we hope that a good
approximation can be achieved with

\begin{equation}
k \ll \textrm{ rank}(A)
\end{equation}

Most of the elements of $A$ are unknown.  But because
it is typically the case that similar users have similar ratings
patterns, we hope to obtain good estimates of all of $A$ even though
only a small part of it is known.

There are two main approaches to matrix factorization in recommender
systems and general machine learning:

\begin{itemize}

\item Singular Value Decomposition (SVD):  As will be seen below, this
is a ``cousin'' of PCA.

\item Nonnegative Matrix Factorization (NMF): Here the matrix $A$ has
nonnegative elements, and one desires the same property for $W$ and $H$.
This may lead to sparsity in the latter two matrices, and in some cases
a nice intepretability.

\end{itemize} 

\section{Notation}

We'll use the following notation for a matrix $Q$

\begin{itemize}

\item $Q_{ij}$:  element in row $i$, column $j$

\item $Q_{i \cdot}$:  row $i$

\item $Q_{\cdot j}$:  column $j$

\end{itemize}

\section{Synthetic, Representative RS Users}

Note the key relation, which we showed in Section \ref{partmat}:

\begin{equation}
(WH)_{i \cdot} = \sum_{m=1}^k W_{im} H_{m \cdot}
\end{equation}

In other words, in (\ref{awh}), we have that:

\begin{itemize}

\item [(a)] The entire vector of predicted ratings by user $i$ can be
expressed as a linear combination of the rows of $H$.

\item [(b)] The rows of $H$ can be thought of as an ``approximate basis'' for
the rows of $A$.

\item [(c)] The rows of $H$ can thus be thought of as synthetic
``users'' who are representative of users in general.  $H_{rs}$ is the
rating that synthetic user $r$ gives item $s$.

\end{itemize} 

In linear algebra terms, the coefficients in (a) above are the
(approximate) \textit{coordinates} of $A_{i \cdot}$ with respect to the
rows of $H$.

In this manner, we can predict for any user that is already in $A$.  but
what about an entirely new user?  What we need is the coordinates of
this new user with respect to the rows of $H$.  We'll see how to get
these later in the chapter.\footnote{After accumlating enough new users,
of course, we should update $A$, and thus $W$ and $H$.}

Of course, interchanging the roles of rows and columns above, we have
that the columns of $W$ serve as an approximate basis for the columns of
$A$.  In other words, the latter become synthetic, representative items.

\section{The Case of Entirely Known A}

This notion of {\it nonnegative matrix factorization} has become widely
used in a variety of ML applications in which the matrix $A$ entirely
known. 

\subsection{Image Classification}

Say we have $n$ image files, each of which has brightness data for $r$
rows and $c$ columns of pixels.  We also know the class, i.e.\ the
subject, of each image, say car, building, person and so on.  From this
data, we wish to predict the classses of new images. Denote the class of
image $j$ in our original data by $C_j$.

We form a matrix $A$ with $rc$ rows and $n$ columns, where the $i^{th}$
column, $A_{\cdot j}$, stores the data for the $j^{th}$ image, say in
column-major order:  This column of $A$ would first store column 1 of
the image, then store column 2 of the image, and so on.\footnote{For
simplicity here we will assume greyscale.  For color, each column of $A$
will consist of three pixel vectors, one for each primary color.}

In the sense stated above, the columns of $W$ typify images, serving as
synthetic, representative images.  Column $j$ of $A$ is then
approximately a linear combination of the columns of $W$, with the
coefficients being the elements of column $j$ of $H$.

So, in predicting the class of a new image, we need to find the
coordinates of that image with respect to the columns of $W$.

\subsection{Text classification}

Here $A$ consists of, say, word counts. We have a list of $d$ key words,
and $m$ documents of known classes (politics, finance, sports etc.).
$A_{ij}$ is the count of the number of times word $i$ appears in
document $j$.

Otherwise, the situation is the same as for image recognition above.  We
find the NMF, and then given a new document to classify, with word
counts $q$, we find its coordinates $q_p$, and then predict class by
matching to the rows of $W$.

\section{NMF}

\subsection{The R Package NMF}

The R package {\bf NMF} is quite extensive, with many, many options.  In
its simplest form, though, it is quite easy to use.  For a matrix {\bf
a} and desired rank {\bf k}, we simply run

\begin{lstlisting}
> nout <- nmf(a,k)
\end{lstlisting}

Here the returned value {\bf nout} is an object of class {\bf "NMF"}
defined in the package.  It uses R's S4 class structure, with
\lstinline{@} as the delimiter denoting class members.  

As is the case in many R packages, {\bf "NMF"} objects contain classes
within classes.  The computed factors are in {\bf nout@fit@W} and {\bf
nout@fit@H}.

Let's illustrate it in an image context, using the following:

\includegraphics[width=3.6in]{Images/MtRush.png}

Here we have only one image, and we'll use NMF to compress it, not do
classification.  First obtain $A$:

\begin{lstlisting}
> library(pixmap) 
# read file
> mtr <- read.pnm('MtRush.pgm') 
> class(mtr)
[1] "pixmapGrey"
attr(,"package")
[1] "pixmap"
# mtr is an R S4 object of class "pixmapGrey"
# extract the pixels matrix
> a <- mtr@grey
\end{lstlisting}

Now, perform NMF, find the approximation to $A$, and display it:

\begin{lstlisting}
> aout <- nmf(a,50)
> w <- aout@fit@W
> h <- aout@fit@H
> approxa <- w %*% h
# brightness values must be in [0,1]
> approxa <- pmin(approxa,1) 
> mtrnew <- mtr
> mtrnew@grey <- approxa 
> plot(mtrnew) 
\end{lstlisting}

Here is the result:

\includegraphics[width=3.6in]{Images/MtRush50.png}

This is somewhat blurry.  The original matrix has dimension $194
\times 259$, and thus presumably has rank 194.\footnote{This is
confirmed by running the function {\bf rankMatrix()} in the {\bf Matrix}
package.} We've approximated the matrix by one of rank only 50, a
storage savings.  Not important for one small picture, but possibly
worthwhile if we have many. The approximation is not bad in that light,
and may be good enough for image recognition or other applications.

\subsection{Algorithms}

How are the NMF solutions found?  What is {\bf nmf()} doing internally?

Needless to say, the methods are all iterative, with one approach being
that of the Alternating Least Squares algorithm.  By the way, this is
not the default for {\bf nmf()}; to select it, set {\bf method =
'snmf/r'}.

\subsubsection{Objective Function}

We need an {\it objective function}, a criterion to optimize, in this
case a criterion for goodness of approximation. Here we will take that
to be the {\it Frobenius} norm, which is just the Euclidean ($L_2$)
norm with the matrix treated as a vector:\footnote{The $L_p$ norm of a
vector $v = (v_1,...,v_r)$ is defined to be

$$
\left (\sum_i |v_i|^p \right )^{1/p}
$$
}

\begin{equation}
\|Q\|_2 = 
\sqrt{
\sum_{i,j} Q_{ij}^2
}
\end{equation}

So our criterion for error of approximation will be

\begin{equation}
\label{errawh}
\|A - WH\|_2
\end{equation}

This measure is specified in {\bf nmf()} by setting {\bf objective =
'euclidean'}.

\subsubsection{Alternating Least Squares}

So, how does it work?  It's actually quite simple.  Suppose just for a
moment that we know the exact value of $W$, with $H$ unknown.  Then for
each $j$ we could minimize

\begin{equation}
\label{errajwhj}
\|A_{\cdot j} - W H_{\cdot j}\|_2
\end{equation}

We are seeking to find $H_{\cdot j}$ that minimizes (\ref{errajwhj}),
with $A_{\cdot j}$ and $W$ known.  But since the Frobenius norm is just
a sum of squares, that minimization is just a least-squares problem,
i.e.\ {\it linear regression}; we are ``predicting'' $A_{\cdot j}$ from
$W$.  The solution is well-known to be\footnote{If you have background
in regression analysis, you might notice there is no constant term,
$\beta_0$, here.}

\begin{equation}
H_{\cdot j} = (W'W)^{-1} W' A_{\cdot j}
\end{equation}

R's {\bf lm()} function does this for us

\begin{lstlisting}
> h[,j] <- lm(a[,j] ~ w - 1)
\end{lstlisting}

for each $j$.\footnote{The -1 specifies that we do not want a constant term in
the model.}

On the other hand, suppose we know $H$ but not $W$.  We could take 
transposes,

\begin{equation}
A' = H' W'
\end{equation}

and then just interchange the roles of $W$ and $H$ above.  Here a call
to {\bf lm()} gives us a row of $W$, and we do this for all rows.

Putting all this together, we first choose initial guesses, say random
numbers, for $W$ and $H$; {\bf nmf()} gives us various choices as to how
to do this.  Then we alternate: Compute the new guess for $W$ assuming
$H$ is correct, then choose the new guess for $H$ based on that new $W$,
and so on.

During the above process, we may generate some negative values.  If so,
we simply truncate to 0.

\subsubsection{Multiplicative Update}

Alternating Least Squares is appealing in several senses.  At each
iteration, it is minimizing a {\it convex} function, meaning in essence
that there is a unique local and global minimum; it is easy to
implement, since there are many least-squares routines publicly
available, such as {\bf lm()} here; and above all, it has a nice
interpretation, predicting columns of $A$.

Another popular algorithm is {\it multiplicative update}, due to Lee and
Seung.  Here are the update formulas for $W$ given $H$ and {\it vice
versa}:

\begin{equation}
W \leftarrow W \circ 
\frac
{AH'}
{WHH'}
\end{equation}

\begin{equation}
H \leftarrow H \circ 
\frac
{W'A}
{W'WH}
\end{equation}

where $Q \circ R$ and $\frac{Q}{R}$ represent \underline{elementwise}
multiplication and division with conformable matrices $Q$ and $R$, and
the juxtaposition $QR$ means ordinary matrix multiplication.

\subsection{Predicting New Cases}

Once we have settled on $W$ and $H$, what can we do with them?  In the
recommender system application mentioned earlier, we simply multiply
them, and then retrieve the predicted values at the matrix positions at
which we did not have user ratings.  But recall the description given
above for the image and text processing examples: 

\begin{quote}
Given the $rc$-element vector $q$ of the new image, we find
its representation $q_p$ in the pseudo-basis, and then find $g$ such
$W_{g \cdot}$ best matches $q_p$.  Our predicted class is then $C_g$.
\end{quote}

How do we find $q_p$?  The answer is that again we can use the
least-squares above.

\begin{lstlisting}
> qp <- lm(q ~ h - 1)
\end{lstlisting}

\subsection{Convergence and Uniqueness Issues}

There are no panaceas for applications considered here.  Every solution
has potential problems.

With NMF, an issue may be uniqueness --- there might not be a unique
pair $(W,H)$ that minimizes (\ref{errawh}).\footnote{See Donoho and
Stodden,{\it When Does Non-Negative Matrix Factorization Give a Correct
Decomposition into Parts?},
\url{https://web.stanford.edu/~vcs/papers/NMFCDP.pdf}.  } In turn, this
may result in convergence problems. The NMF documentation recommends
running {\bf nmf()} multiple times; it will use a different seed for the
random initial values each time.

The Alternating Least Squares method used here is considered by some to
have better convergence properties, since the solution at each iteration
is unique.  

\subsection{How Do We Choose the Rank?}

This is not an easy question.  One approach is to first decompose our
matrix $A$ via {\it singular value decomposition}.  Roughly speaking,
SVD finds an orthonormal basis for A, but then treats the vectors in
that basis as random variables.  The ones with relatively small variance
may be considered unimportant.

\begin{lstlisting}
> asvd <- svd(a)
> asvd$d
  [1] 1.188935e+02 2.337674e+01 1.685734e+01 1.353372e+01 1.292724e+01
  [6] 1.039371e+01 9.517687e+00 9.154770e+00 8.551464e+00 7.776239e+00
 [11] 6.984436e+00 6.505657e+00 5.987315e+00 5.059873e+00 5.032140e+00
 [16] 4.826665e+00 4.576616e+00 4.558703e+00 4.107498e+00 3.983899e+00
 [21] 3.923439e+00 3.606591e+00 3.415380e+00 3.279176e+00 3.093363e+00
 [26] 3.019918e+00 2.892923e+00 2.814265e+00 2.721734e+00 2.593321e+00
\end{lstlisting}

So, the first standard deviation is about 119, the next about 23 and so
on.  The seventh is already down into the single-digit range.  So we
might take $k$ to be, say, 10.  The columns of $A$ lie in a subspace of
$R^{194}$ of dimension of about 10.

\subsection{Why Nonnegative?}

In the applications we've mentioned here, we always have $A \geq 0$.
However, that doesn't necesarily mean that we need $W$ and $H$ to be
nonnegative, since we could always truncate.  Indeed, we could consider
using SVD instead.

There are a couple of reasons NMF may be preferable.  First, truncation
may be difficult if we have a lot of negative values.  But the second
reason is rather philosophical, as follows:

In the image recognition case, there is hope that the vectors $W_{\cdot
j}$ will be {\it sparse}, i.e.\ mostly 0s. Then we might have, say, the
nonzero elements of $W_{\cdot 1}$ correspond to eyes, $W_{\cdot 2}$
correspond to nose and so on with other parts of the face.  We are then
``summing'' to form a complete face.  This may enable effective {\it
parts-based recognition}.

Sparsity (and uniqueness) might be achieved by using {\it
regularization} methods, in which we minimize something like

\begin{equation}
\|A - WH\| + \lambda (\|W\|_1 + \|H\|_1)
\end{equation}

where the subscript 1 (``$L_1$ norm'') means the norm involves sums of
absolute values rather than sums of squares.  This guards against one of
the factors becoming ``too large,'' and it turns out that this also can
lead to sparse solutions.  We can try this under {\bf nmf()}  by setting
{\bf method = 'pe-nmf'}.


