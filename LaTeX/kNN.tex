\chapter{Nearest-Neighbor Methods}  
\label{chap:knn} 

Recall our preview in Chapter \ref{chap:prologue}.
\textit{Collaborative filtering} RS methods are based on known user-item
ratings, as opposed to, e.g., methods based on textual user evaluations.
All of the methods in that chapter fall into this category.  The most
natural, and earliest, such method was \textit{user-based filtering},
also known as \textit{memory-based filtering}.

\section{Notation}

As before, let $A$ denote the ratings matrix.  The element $a_{ij}$ in
row $i$, column $j$, is the rating that user $i$ has given/would give to
item $j$. In the latter case, $a_{ij}$ is unknown, and its predicted
value, to be determined, will be denoted by $\widehat{a}_{ij}$.
Following R notation, we will refer to the unknown values as NAs.

Note that for large applications, the matrix $A$ is far too large to
store in memory.  One could resort to storage schemes for
\textit{sparse} matrices, e.g.\ \textit{Compresed Row Storage}, but here
we will simply use $A$ to help explain concepts.  In the
\textbf{rectools} package, the input data is run through
\textbf{formUserData()}, and algorithms use that instead of $A$.  This
function organizes the data into an R list, one element per user.  Each
such element records the ratings made by that user. 

Let's refer to a new case to be predicted as NC.

\section{User-Based Filtering}

In predicting how a given user would rate a given item, we first find
all users that have rated the given item, then determine which of those
users are most similar to the given user.  Our prediction is then the
average of the ratings of the given item among such ``similar'' users.
A corresponding approach based on items is used as well.  We focus on
such methods in this chapter.

\subsection{Matrix View}
\label{matrixview}

In terms of the matrix $A$ above, we first look at column $j$.  We cull
the rows having non-NA values in this column, and for each of those
rows, find the distance of that row to the NC.  Take the $k$ closest
rows, and finally, average the value of $a_{mj}$ to get
$\widehat{a}_{ij}$ (with $m$ ranging over the row numbers of the
selected rows).

\subsection{(One) Implementation}

Below is code from \textbf{rectools} (somewhat
simplified).\footnote{Code written mainly by Vishal Chakraborti.}
The
arguments are:

\begin{itemize}

\item \textbf{origData:}  The original dataset, after having been run through
\textbf{formUserData()}. 

\item \textbf{newData:}  The element of \textbf{origData} for NC.\footnote{If
NC is new, not in the database (called \textit{cold start}), we
synthesize a list element for it, assuming NC has rated at least one
item.  We set the user ID for NC to ''.}

\item \textbf{newItem:}  ID number of the item to be predicted for NC.

\item k:  The number(s) of nearest neighbors.  Can be a vector, i.e.\ we
can request several predictions to be made at once.

\end{itemize} 

\begin{lstlisting}[numbers=left]
predict.usrData <- function(origData,newData,newItem,k)
{
   # we first need to narrow origData down to the users who 
   # have rated newItem

   # here oneUsr is one user record in origData; the function will 
   # look for a j such that element j in the items list for 
   # this user matches the item of interest, newItem; (j,rating) 
   # will be returned

   checkNewItem <- function(oneUsr) {
      whichOne <- which(oneUsr$itms == newItem)
      if (length(whichOne) == 0) {
         return(c(NA,NA))
      } else return(c(whichOne,oneUsr$ratings[whichOne]))
   }

   found <- as.matrix(sapply(origData,checkNewItem))
   # description of 'found':
   # found is of dimensions 2 by number of users in training set
   # found[1,i] = j means origData[[i]]$itms[j] = newItem;
   # found[1,i] = NA means newItem wasn't rated by user i
   # found[2,i] = rating in the non-NA case

   # we need to get rid of the users who didn't rate newItem
   whoHasIt <- which(!is.na(found[1,]))
   origDataRatedNI <- origData[whoHasIt]
   # now origDataRatedNI only has the relevant users, the ones who 
   # have rated newItem, so select only those columns of the found 
   # matrix
   found <- found[,whoHasIt,drop=FALSE]

   # find the distance from newData to one user y of origData; 
   # defined for use in sapply() below
   onecos <- function(y) cosDist(newData,y,wtcovs,wtcats)
   cosines <- sapply(origDataRatedNI,onecos)
   # the vector cosines contains the distances from newData to all the
   # original data points who rated newItem

   # action of findKnghbourRtng(): find the mean rating of newItem in
   # origDataRatedNI, for ki (= k[i]) neighbors
   #
   # if ki > neighbours present in the dataset, then the 
   # number of neighbours is used
   findKnghbourRtng <- function(ki){
     ki <- min(ki, length(cosines))
     # nearby is a vector containing the indices of the ki closest 
     # neighbours
     nearby <- order(cosines,decreasing=FALSE)[1:ki]
     mean(as.numeric(found[2, nearby]))
   }
   sapply(k, findKnghbourRtng)
}
\end{lstlisting}

Note that the distances were computed by the function
\textbf{cosDist()}, which computes a ``cosine'' similarity (not really a
distance).  It could be replaced by a function of the analyst's choice.
More on this below.\footnote{An R note: \textbf{sapply()} is called
several times.  Since that function needs another function, say
\textbf{f()}, as an argument, there is a question as to where to define
that function --- inside \textbf{predict.usrData()} or outside it.

If \textbf{()} is short, it arguably is clearer to define it
``inline,'' i.e.\ just before the call to \textbf{sapply()}.  In
addition, we may need fewer arguments to \textbf{f()}.  In the above
code for instance, consider the function \textbf{findKnghbourRtng()}.
It makes use of \textbf{cosines}, which was computed a few lines earlier
with \textbf{predict.usrData()}.  That variable, while local to the
latter function, is global to \textbf{findKnghbourRtng()}.}

\section{Regression Analog}

Recall the method of k-nearest neighbor (kNN) regression estimation from
Chapter \ref{chap:infra2}, involving prediction of weight from height
and age:

\begin{quote}
To estimate $E(W ~| H=70, A=28)$, we could find, say, the 25 people in our
sample for whom $(H,A)$ is closest to (70,28), and average their weights
to produce our estimate of $E(W ~| H=70, A=28)$.
\end{quote}

So kNN RS is really the same as kNN regression.

Note that kNN is an example of \textit{smoothing}.  In an image
processing tool, one can ``smooth out'' blemishes in a photograph by
replacing the value of each pixel by the average value of neighboring
pixels.  With kNN, we are averaging the ratings if a neighborhood of NC.

\section{Choosing k}

As we have already seen with RS, regression and machine learning methods, 
the typical way to choose a model is to use cross-validation.  This is
true for kNN RS as well; we can choose the value of $k$ via
cross-validation.  However, a curious property may streamline the
process, as follows.

Note this line in \textbf{predict.usrData()}:

\begin{lstlisting}
ki <- min(ki, length(cosines))
\end{lstlisting}

What is happeing here?  Say for a particular situation, there are 52
rows in $A$ having ratings for the item specified in NC.  Then of course
as we increase our number of nearest neighbors beyond 52, we will get
identical results.  Such a situation will occur if the $A$ does not have
a high proportion of non-NA values.  

Indeed, it often turns out that the best value of $k$, i.e.\ the one
that predicts the holdout set the best, is the largest possible value as
above.  In choosing $k$, then, we might start our search as
\textbf{length(cosines}), then try a few values slghtly smaller than
that.  If the latter give slightly less accuracy, we might choose our
$k$ at that right endpoint.

\section{Defining Distance}

As mentioned, we could use any distance function instead of
\textbf{cosDist()} above to measure the similarity of NC and rows in
$A$.  Many distance measures have been proposed, but typically some kind
of ``cosine'' method is chosen.\footnote{One could use ordinary
Euclidean distance as in kNN regression, but the cosine methods are
preferred because they take values in [0,1], regardless of the lengths
of the ratings records being compared.} What does this mean?

\subsection{The Basic Cosine}

In 2- or 3-dimensional space, we are familiar with the cosine of the
angle between vectors $x$ and $y$:

\begin{equation}
\cos(x,y) = \frac{(x,y)}{||x|| ~ ||y||}
\end{equation}

where the numerator is the familiar ``dot product.''  But even in higher
dimensions --- something like the number of columns of $A$, which could
be quite large --- the above expression makes sense.  It is bounded by
-1 and 1 in general, and since the elements of $A$ are nonnegative, the
cosine is bounded between 0 and 1, no matter what dimension we are in.
As such, it would seem to be a good similarity measure.\footnote{It is
not actually a distance, as it does not satisfy the Triangle Inequality,
$\textrm{dist}(x,z) \leq \textrm{dist}(x,y) + \textrm{dist}(y,z)$.}

Here is (a slightly simplified version of) \textbf{cosDist()}:

\begin{lstlisting}
cosDist <- function(x,y)
{
   # rated items in common
   commItms <- intersect(x$itms,y$itms)
   if (length(commItms)==0) return(NaN)
   # where are those common items in x and y?
   xwhere <- which(!is.na(match(x$itms,commItms)))
   ywhere <- which(!is.na(match(y$itms,commItms)))
   xvec <- x$ratings[xwhere]
   yvec <- y$ratings[ywhere]

   xvec %*% yvec / (l2a(xvec) * l2a(yvec))
}

l2a <- function(x) sqrt(x %*% x)

\end{lstlisting}

\subsection{Refinements of Cosine}

Cosine is used in many areas of machine learning, notably text
classification.  The analog there of our $A$ matrix is the
\textit{document term matrix}.  One develops a list of words that are
found to predict text category well, e.g.\ sports, business, politics and
so on.  The matrix $A$ would then be of size $w \times d$, where $w$ is
the number of words in our list and $d$ is the number of documents.  The
$(i,j)$ element is, e.g., 1 or 0, depending on whether that word appears
in that document.  Similarity of two documents is measured by the
cosine between the $w$-word vectors for the two documents.  When a new
document comes in of unknown category, we find the $k$ documents in our
training data that are most similar to the new one, and determine which
category is most frequent among these $k$ known documents.  That
category is then our guess for the new case.

\subsubsection{Correlation}

In our RS context, though, pure cosine might be a less appropriate
measure of similarity.  Say for instance we have users U and V, the
first of whom is rather liberal in her ratings while V is more cautious.
It's still possible that the two are similar, in the predictive sense,
if they are highly correlated:  Whenever U gives an item a rating
above U's average, typically V rates the item more highly than V's
average.  In that case, we should use correlation as our measure of
similarity:

\begin{equation}
\textrm{corr}(x,y) =
\frac{(x-\overline{x},y-\overline{y})}
{||x-\overline{x}|| ~ ||y-\overline{y}||}
\end{equation}

Here $\overline{x}$ is the vector having all of its elements equal to
the average value in $x$, with $\overline{y}$ defined similarly.

\subsubsection{Jaccard Similarity}

Suppose User A's ratings are always gives exactly double those of User
B.  Then most cosine-based ratings (there are others besides the
above) will say that A and B are exactly similar.  Yet if we predict the
rating for some item for User B from A's rating, it will be twice as
high as a it should be.  An alternative is Jaccard similarity.

Here again, let $x$ and $y$ be two rows in $A$.  Define
$\widetilde{x}$ to be the indices, i.e.\ column numbers of the known
elements of $x$; it's simply a list of the IDs of the items rated by
this user.  Define $\widetilde{y}$ analogously.  Finally, for any set
$B$, let $|B|$ denote the \textit{cardinality} of the set, meaning the
number of elements.  Then the Jaccard similarity between $x$ and $y$ is
defined to be

\begin{equation}
\frac{|\widetilde{x} \cap \widetilde{y}|}
{|\widetilde{x} \cup \widetilde{y}|}
\end{equation}

For the MovieLens data, for example, this is measuring whether these two
users are rating the same movies (with the denominator being a scaling
factor that places the measure in [0,1]).  We are not considering the
actual ratings of the movies, just the degree to which the two users
have the same tastes.

By the way, this brings up a very important point about RS in general.
Typically someone will watch (and rate) a movie with the expectation
that they'll like it.  In this sense, the ratings may be biased, and our
guess for the rating the user would give to a movie he does not plan to
watch may be shaky. 

\subsubsection{Forming Neighborhoods Using Matrix Factorization}

An obvious drawback to cosine methods is that they are based on small
sample sizes.  The number of users who have rated the requested item may
be too small to obtain good accuracy.  An alternative using matrix
factorization (MF) is as follows.

Recall that MF methods calculate matrices $W$ and $H$ such that $A
\approx WH$.  As will be shown in Chapter \ref{chap:svd}, the rows of
$H$ can be thought of as a set of ``representative'' synthetic users,
approximately spanning the space of all users.  Row $i$ of $W$ is the
set of coefficients of user $i$ with respect to this basis.  Note that
the MF estimate of $a_{ij}$, $\widehat{a}_{ij}$, is the dot product of
row $i$ in $W$ and column $j$ in $H$.  And thus the product of row $i$
of $W$ with all of $H$ is the vector of all MF  predictions for user $i$,
across all items.

Now consider what will happen with NC.  Assume for now that NC is a user
already in the database, row $r$,  and let the item of interest to NC be
item $s$.  We can now form a neighborhood of NC by finding the $k$ most
similar rows of $W$ to row $r$.  Then for each of these rows $m$, find
$\widehat{a}_{ms}$, then average those values to obtain our hybrid
MF-kNN prediction for NC. 

Note that MF is a form of smoothing, as is kNN, so this hybrid amounts
to double-smoothing.  

\textit{Top-N Recommendations}

Taking that hybrid idea one step further, say we multiply those $m$ rows
times $H$, and then average the resulting rows.  This gives us hybrid
MF-kNN predictions for user $r$ for all the items.  In some RS
applications, we want the top-N of these, which are easy to obtain here.

\section{Item-Based Filtering}

Here we modify the description in Section \ref{matrixview}.  Given a
query about item $s$ from NC, user $r$, we find the $k$ closest columns
to column $s$, among those containing a rating by the given user.  We
then average those ratings.

\section{Covariates}

In gauging the similarity between two users, we may wish to make use of
related variables, known variously as \textit{covariates}, \textit{side
information} or \textit{attributes}.  This is easily implemented in kNN
RS.  In computing the similarity between users A and B, instead of
merely comparing their ratings vectors, we extend those vectors with the
covariates information, and compute similarity between the extended
versions.

Let's see how this is done in \textbf{rectools}.  The call form for
\textbf{formUserData()} is

\begin{lstlisting}
formUserData(ratingsIn, usrCovs = NULL, itmCats = NULL)
\end{lstlisting}

Here the first argument is as before, and both the second and third
arguments are for covariates.  The argument \textbf{usrCovs} is intended
for things like user age, gender and so on.  The argument
\textbf{itmCats} is for content categories, a vector of
proportions.\footnote{The two arguments could have been replaced by one,
but it was desired to give the analyst the ability to assign different
weights to them, as seen below.}  In the MovieLens data, category $r$
would be the proportion of this user's ratings that are for movies in
genre $r.$\footnote{Recall that these may sum to more than 1.0.} Here
the second and third arguments are data frames, one row per
user.  The argument \textbf{usrCovs} becomes the \textbf{cvrs} component
for that user in the return value of this function, with
\textbf{itmCats} becoming the component \textbf{itms}.

The full call form for prediction is then

\begin{lstlisting}
predict.usrData(origData,newData,newItem,k,wtcovs=NULL,wtcats=NULL)
\end{lstlisting}

The final two arguments are weights to be placed on the two.  These
control how much weight they should carry in the similarity computation,
relative to the ratings.

Here is some more of the code in the function \textbf{cosDist()}:

\begin{lstlisting}
if (!is.null(wtcovs)) {
    xvec <- c(xvec, wtcovs * x$cvrs)
    yvec <- c(yvec, wtcovs * y$cvrs)
}
\end{lstlisting}

The reader can see the vectors being extended by the covariates, the
latter first being multiplied by the weights.


\iffalse

idxs <- sample(1:nrow(ml),1000)
ml <- ml[,1:3]
mltrn <- ml[-idxs,]
mltst <- ml[idxs,]
library(rectools)
ud <- formUserData(mltrn)

k <- 5
z <- apply(mltst[,-3],1,pn); mean(abs(z-mltst[,3]),na.rm=T)

predname <- function(userData,oneInputRow,k)
{
   usr <- oneInputRow[1]
   itm <- oneInputRow[2]
   udElement <- userData[[as.character(usr)]]
   predict(userData,udElement,itm,k)
}

pn <- function(oneInputRow) predname(ud,oneInputRow,k)


\fi

